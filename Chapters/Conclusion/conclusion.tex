% !TeX root = ../../main.tex
%The conclusion is a key part of the dissertation. It should be the natural end-point of
%the flow of argument that started from the Introduction with the identification of the
%problem to be tackled. It should draw upon the problem description and the literature
%survey, using the results to compare the work done with the expected outcome and
%with previous or related work. It should draw together the lessons learnt from your
%critical evaluation of the development/ implementation/ research processes used within
%the project. It should lead to a set of natural conclusions about the achievements made
%and the ways in which the difficulties encountered would be tackled if the project were
%to be run again. It should clearly identify any contributions to current knowledge or
%practice that the project provides. The conclusion should therefore be the culmination
%of the project, highlighting the achievements of the project, identifying things that have
%been done well, noting areas which could have been done better or tackled in a different
%way, and relating the work of the project to your intended system (as detailed in the
%Introduction) and related work (as discussed in the Literature Survey).
\chapter{Conclusion}
\label{chap:conclusion}