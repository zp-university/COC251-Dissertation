% !TeX root = ../../main.tex
%The conclusion is a key part of the dissertation. It should be the natural end-point of
%the flow of argument that started from the Introduction with the identification of the
%problem to be tackled. It should draw upon the problem description and the literature
%survey, using the results to compare the work done with the expected outcome and
%with previous or related work. It should draw together the lessons learnt from your
%critical evaluation of the development/ implementation/ research processes used within
%the project. It should lead to a set of natural conclusions about the achievements made
%and the ways in which the difficulties encountered would be tackled if the project were
%to be run again. It should clearly identify any contributions to current knowledge or
%practice that the project provides. The conclusion should therefore be the culmination
%of the project, highlighting the achievements of the project, identifying things that have
%been done well, noting areas which could have been done better or tackled in a different
%way, and relating the work of the project to your intended system (as detailed in the
%Introduction) and related work (as discussed in the Literature Survey).
\chapter{Conclusion}
\label{chap:conclusion}

Overall, I believe the smart lock system has been implemented to a very high standard, the minimum requirements for the system were met, and the technical and user aims and objectives were met.

There were plenty of high and low points throughout this project, however I think the biggest high point was when I got the demo working finally after having to work through what seemed like a never ending list of bugs. Getting the project to that point was a great achievement as it took a lot of work.

My biggest strength throughout the project was my ability to encounter technical problems and work through them, coming up with a solution that wouldn't just work but still maintained the aims of the project and satisfied the requirements that I had set out at the very beginning. Alongside this, I believe my use of version control was another strength as it allowed me to keep track of changes with detailed commit messages on what was changed allowing me to easily identify where a certain bug could have been introduced.

On the downside, my time management was not the best, and that coupled with a lot of other events taking place in my life meant that I didn't have enough time to work on some of the optional requirements that I would have liked to include in the project to improve the usefulness of the product and further prove that security and usability can work hand in hand.

However despite this I believe this project has provided proof that a secure smart lock can be created, and that if developed further into a real product this current system, coupled with the extra ideas provided in the design section that weren't implemented, could rival some of the smart locks out there on the market right now.