% !TeX root = ../../main.tex
\chapter{Requirements}
\label{chap:requirements}

\section{Requirements Specification}

\subsection{Minimum Requirements}
The following requirements are what must be implemented in the system prior to submission of the project in order to provide a minimal viable project (MVP) that could actually be connected to motor and used in a house:

\begin{itemize}
	\item System should indicate whether door is unlocked or locked with LEDs
	\item Seperate Raspberry Pi's should be used to show system modularity (i.e. core, door lock and nfc)
	\item Core should provide an API for other modules to use so that making new modules is easy
	\item Remote server should be setup with an API and database for a central point of trust
	\item All communication should be secured and tested against common vulnerabilities
	\item An Android app should be created to make setup and configuration of the system easy for an inexperienced user
	\item It should be as easy or easier to use this system than to use a normal key lock system
	\item More than just the owner of the lock should be able to have the app and unlock the house
\end{itemize}

\subsection{Extra Requirements}

If extra time is available at the end of the project, I would like to implement extra features that would expand the project from an MVP into something more feature rich and useful for an everyday user. Some of those features may include the following:

\begin{itemize}
	\item Allow certain people to only enter the house at set times through the Android app
	\item Mount the system to a real door and have a motor unlock the door automatically
	\item Include a web interface so that people can manage their account and lock from the web panel
	\item Allow more than one lock system to be used per account for people who want multiple locks in a house or have more than one property
	\item Implement automatic unlocking when a phone gets close enough to the door
	\item Get alerted if people are still in the house after their allotted access period has expired
	\item Make extra modules for the lock such as a face recognition module, a doorbell module, etc
\end{itemize}


\section{Risk Management}
For a project such as this, there is obviously a large amount of risk involved if for some reason the system isn't secured correctly. My plan is to limit these risks as much as possible by ensuring that all communication is using enforced TLS 1.2 as well as not running the system on an actual door until I have a working system that I have run penetration testing on. Replay attacks were something that I had to consider, but these are averted by using TLS 1.2, as it has protection against that built in using the MAC secret and the sequence number. When running this system on a real door, there will be much larger risk involved as it is obviously securing a real house, however as long as all the proper testing is done before it gets installed on a real door, then that should eliminate the risk involved. Obviously, if this gets installed on any door, then it will be in my own house, so I will ensure that the system is functioning properly before installation and any updates are fully tested before being installed on the live system.

