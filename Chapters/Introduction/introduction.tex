%This section will be relatively short. It should introduce the reader to the problem that
%is to be tackled, and provide a brief indication of the context within which the problem
%exists. This will include some consideration of appropriate related literature and/or
%past attempts at solving the problem, although the main consideration of previous work
%should be left until the Literature Survey. The most important part of the introduction
%is the identification of why the problem is of particular interest and worthy of further
%study. The introduction should seek to draw the reader into the remainder of the
%dissertation, whetting their appetite to learn more of the problem and its solution. It
%should introduce the structure of the document and provide a framework for the reader
%as progress is made through the remainder of the dissertation.
\chapter{Introduction}
\label{chap:Introduction}

%\section{Problem Formulation}
%
%\section{Aims and Objectives}
%
%\section{Outline of Areas of Research}
%
%\subsection{Sub Section 1}
%
%\subsection{Sub Section 2}
%
%\subsubsection{Sub Sub Section 1}
%
%\subsubsection{Sub Sub Section 2}
%
%\subsubsection{Sub Sub Section 3}
%
%\subsubsection{Sub Sub Section 4}
%
%\subsection{Sub Section 3}
%
%\section{Contribution of Thesis}
%
%\section{Thesis Outline}