% !TeX root = ../../main.tex
\chapter{Literature Review}
\label{chap:literature_review}

\section{Aim}
Investigate different aspects of a secure smart lock system to see where potential security issues could be and how to avoid or counteract them.
\section{Objectives}
\begin{itemize}
	\item {Evaluate different areas of the Secure Smart Lock System}
	\item Investigate methods of ensuring the security of the system
	\item Investigate ways to avoid potential security risks imposed by necessary parts of the system
	\item Ensure that security doesn't make the system considerably more difficult for the user to setup, use day-to-day or manage
\end{itemize}

\section{Current Smart Locks}

\subsection{Overview}
Smarts locks and internet of things (IoT) devices, in general, have increased massively in availability and use over the last couple of years and according to \cite{Buhov2016}, there will be 50 billion devices connected to the internet by 2020. Having these devices in and around your home constantly connected to the internet introduces some risk if they are not properly secured. This risk is exaggerated if they are devices that secure your home such as a smart lock.
\\
\indent\cite{Ho2016} compare 5 commonly available smart locks that are on the market right now. These locks operated in three different ways: touch-to-unlock, mobile app unlocking, and automatic unlocking. For my purposes, I am mostly interested in the mobile app unlocking as it will be the main way to unlock the Secure Smart Lock System. All the smart locks except for the Lockitron connect to the internet through a Bluetooth connection to the phone, rather than connecting directly, this is interesting as my system will only use Wi-Fi, and will only communicate directly with the phone if they are on the same Wi-Fi network.  The study concluded that all the smart locks except for the Lockitron were vulnerable to the attacks that they undertook, this was down to the Lockitron using Wi-Fi directly and therefore being able to verify with the server for the most up-to-date access control lists. They continue to explain that direct connections, while good, do mean that the device must include a Wi-Fi module which increases power usage on the device and so battery-only operation isn't feasible. A good point they make about the Lockitron is that it doesn't maintain its own access control list locally, so in the event of their servers being unavailable, the lock will not function. This is something I plan to avoid in my lock by maintaining a local access control list, however, due to the attacks mentioned in the study, I will ensure that any changes to the access control list are passed onto the lock before the user is told they are saved.
\\
\indent\cite{Ye2017} did a study specifically on the security of the August Smart Lock. This study focussed on more specific attacks such as moving the owner account from one phone to another and taking control of it. All of the attacks that they performed required root access to the Android device in order to perform, which is unreasonable to assume you will be able to get without taking the users device and knowing their unlock code to perform the rooting process. This is further shown by \cite{Fuller2017} who also tried to penetrate the August Lock but without using rooting as their main point of entry. The attacks that \cite{Ye2017} performed certainly aren't ideal, but rooting compromises the integrity of the device and exposes all the data on the device including encryption keys, app data and anything else the device has stored. As shown by \cite{Fuller2017} the August Lock has had a lot of past security issues, however, they have been fixed by the development team since then.

\subsection{The August Lock}
\subsubsection{Overview}
The August lock is one of the most well known locks on the market, it has a wide variety of different features and comes with support for both Android and iOS devices. One of the major benefits of the august lock is you can fit it to your existing door lock so installation required is minimal. There are two different models of the August Lock, their original model and their new "Smart Lock Pro".

\subsubsection*{Significant Features}
\begin{itemize}
	\item Mobile phone is now your key
	\item You can allow other people access to your property with virtual keys and can also set time constraints on each virtual key
	\item Keep existing door hardware, don't have to replace lock and you can still use original keys if you want to
	\item Has an Android and iOS app for controlling the lock so most people can use it
	\item Works with Apple HomeKit, Amazon Alexa, Google Home, Nest and IFTTT
	\item Can automatically unlock and lock the door when you enter and leave the house
	\item DoorSense allows the lock to detect if the door is left ajar
\end{itemize}
\improvement{Figure out how to put sources down for this section}

\subsubsection*{Auto Unlocking/Locking}
According to \cite{August2017}, this feature works by using the bluetooth on your phone along with GPS and Wi-Fi to detect when you are in range of the door and then unlocks it for you over bluetooth. When you are more than 200m away from your house, the phone sits in what they call "away mode" which means it is no longer searching for the lock over bluetooth until the phone gets near to the house again. The problem with this kind of system is that bluetooth is vulnerable to relay attacks, and if the phone is using Wi-Fi to detect its location then according to \cite{Feng2014} it's a fairly trivial attack to spoof the devices location, therefore activating the bluetooth search for the lock allowing for bluetooth relay attacks on the app and the lock. Aside from attacks, the feature itself has other issues such as not automatically locking until you are quite a distance from your lock, in the case of \cite{So2017} the smart lock was still unlocked until she was 1/4 mile from her lock. This is plenty of time for someone else to walk into your house if you live in a built-up area.

\subsubsection*{DoorSense}
DoorSense allows the lock to send you a notification if your door is left open for more than a certain amount of time. This is a feature that will be rarely useful as most people don't leave their door ajar when they enter or leave the house, but in the odd occasion that you do, it would be good to know. However, as mentioned by \cite{So2017}, this feature definitely has some issues, after installing the DoorSense sensor into the doorframe and enabling notifications if the door is left open for one minute, she received no notifications. She then checked it manually by leaving the door open and waiting for notifications on her phone, but got nothing. When working however, the DoorSense feature does integrate with the smart lock directly and allow for instant auto locking when the door is closed which alleviates some of the issues mentioned above.

\subsubsection*{Conclusion}
The August Smart Lock Pro + Connect is a package with a lot of features, however some of them don't work as well as they should, and coming to rely on them may one day result in your house being left unlocked if they don't work as expected. The product has good integration with other technologies such as Apple HomeKit, Amazon Alexa and Google Home, however there's no API available for you to add your own integrations into the system which would be a nice addition.

\subsection{Kwikset Kevo}
\subsubsection*{Overview}
Kwikset are a well known lock manufacturer who have jumped on the smart lock bandwagon. The Kevo lock has multiple different designs which allow you to choose one which suits your home. On the technology side it isn't slacking either coming with integrations into multiple different smart home technologies and the ability to add remote access with the kevo plus, much like the august connect does.

\subsubsection*{Significant Features}
\begin{itemize}
	\item Mobile phone is now your key
	\item You can allow other people access to your property with virtual keys and can also set time constraints on each virtual key
	\item Keep existing door hardware, don't have to replace lock but has different options if you want an entirely new lock
	\item Has an Android and iOS app for controlling the lock so most people can use it
	\item Works with Amazon Alexa, Nest and IFTT
	\item Touch the lock to open when registered device is in range, auto unlock is also available in the app
	\item Can purchase a fob that allows keyless entry without a smartphone
\end{itemize}

\subsubsection*{Auto Unlocking/Locking}
This lock allows auto unlocking through a setting in the app, however auto locking has to be enabled by changing a dip switch on the lock itself which might not be obvious to a normal user. The auto unlocking is done by communicating with a phone that is within range and verifying that it has a key corresponding to the lock, but also checking that the phone is outside the house rather than inside to ensure security. Unlike the August lock, the auto unlock does require a physical touch on the lock in order for it to detect the phone and unlock the door. According to \cite{DeLooper2018} this process of unlocking takes a few seconds, so you don't have the convenience of walking up to your door and it already being open for you. The auto locking is also worse than the August lock as it isn't based on proximity, it's just based on a simple timer, however it means you know that the lock will definitely be locked after a set time of you leaving, which would add to your peace of mind and increases security.

\subsubsection*{Kevo Fob}
The Kevo Fob allows you to have keyless access to the house without having to use your smartphone. This is something that the August lock does not offer and would be great for anyone who doesn't have a smartphone such as older relatives. This is something I plan to have in my system time permitting as it would be good to allow entry to the property without a traditional key but also without requiring the use of a smartphone.

%\subsubsection*{Smart Home Integration}
%Potential subsubsection, this lock has less integrations than other locks do

\subsubsection*{Conclusion}
The Kwikset Kevo + Kevo Plus is less feature filled than the August smart lock, but comes from a much more well known company in the world of locks. The lock lacks integrations with other products compared to the likes of the August Lock, but I couldn't find any published security flaws with the Kwikset Kevo compared to the large amount that are published for the August Lock.

\subsection{Specific Lock 3 Detail}
\info{Choose a lock and write a detailed analysis of it from the webpage and official sources released by the company}

\subsection{Specific Lock 4 Detail}
\info{Choose a lock and write a detailed analysis of it from the webpage and official sources released by the company}

\subsection{Specific Lock 5 Detail}
\info{Choose a lock and write a detailed analysis of it from the webpage and official sources released by the company}

\section{Certificate Pinning}
Certificate pinning is one of many ways to ensure you are talking to the server you think you are. If you end up doing this badly, however, you can decrease the security of your app. A study performed by \cite{Buhov2016} does an evaluation of 25,000 android apps on the play store to see how many of them implement security properly. It turns out that only 21\% of applications have no issues, with 36\% having a broken implementation of either the Trust Manager or the Hostname Verifier. This issue is brought about by people using self-signed certificates or pinning certificates and trying to, therefore, implement their own version of the Trust Manager and/or Hostname Verifier but doing so incorrectly. As pointed out by the study, many of the incorrect implementations of the Trust Manager lead to the app accepting any certificate it was given, therefore breaking the chain of trust that is usually in place on the device. \cite{Tendulkar2014} back this up showing that out of the apps they tested, only 43\% use SSL verification correctly with the remaining apps either accept all certificates or accept all hostnames. An even more interesting fact that they discovered was that 53\% of the apps they tested implement this custom code to allow all certificates or all hostnames even though they are using valid and signed certificates. The inclusion of this incorrect code allows for man-in-the-middle attacks which would compromise the security of user data being transferred over the network.
\\
\indent\cite{Buhov2016} mentions the different security you get from different levels of certificate pinning. You can pin the end (leaf) certificate which tells you with absolute certainty that you are connecting to the server you think you are. One downside of this is that you must make updates to your app frequently as most certificates only have a lifetime of a year. This problem is made even worse with LetsEncrypt who's certificates expire after a maximum of 3 months. The study explains other solutions for certificate pinning including intermediate and root certificate pinning. Intermediate certificate pinning is less secure than leaf certificate pinning, however, as long as you trust the Certificate Authority (CA) you are using to not sign a certificate for your domain erroneously, then it is just as secure. The major benefit of intermediate certificate pinning is that they update a lot less frequently, the current LetsEncrypt certificate doesn't expire until the year 2021. The last option of root certificate pinning is a lot less secure as you are trusting a lot more parties as potentially multiple CA's will use the same root certificate for signing their SSL certificates.
\\
\indent The studies show me that certificate pinning is overall a good idea, however, it must be implemented correctly. Based off of the tests that were performed by these studies I will ensure that my app only accepts certificates signed by the LetsEncrypt intermediate certificate and that my app ensures the hostname on the certificate matches the hostname on the certificate. If either of these tests fail then it means I have implemented the security incorrectly and as these studies have shown that would make my app vulnerable to MITM attacks which would severely impact the security of my smart lock system.

\section{Avoiding NFC Relay Attacks}
During my research it became apparent to me that NFC has no way of verifying that the device it is communicating with is actually right next to it, making it subject to relay attacks \citep{Francis2010}. This study explains the relay attack as the ability to send the NFC communication over a separate communication channel using other phones as the proxy. This introduces problems if the app is always able to receive NFC communication and unlock the door, as an attacker could then unlock the door by having someone by the door NFC reader and also by a person with permission to unlock the door and just send the data between the two attackers. \cite{Oh2015} suggests distance bounding as a potential solution to this issue as communicating between the proxy devices will add a delay that would normally not be present in the system. The NFC reader should be able to determine the distance between the NFC device and the reader by determining the latency in sending and receiving data. The study points out that this kind of verification can vary massively in latency and therefore you may sometimes end up rejecting legitimate users if their device doesn't respond quickly enough. Two-factor Authentication is another solution that the study suggests and is a much more reliable solution as you can request that as well as the NFC card a user must input a pin, provide a fingerprint or enter a password. Another suggestion they make to directly counter relay attacks is to emit a jamming signal around the reader so that the device can't communicate externally via 3G, 4G, Wi-Fi or Bluetooth. Whilst this is an interesting solution, there are many situations where this would be impractical or illegal to implement a jammer in an NFC enabled device.
\\
\indent\cite{Francis2010} shows how trivial it is to relay NFC data across a Bluetooth connection even on old devices. This study shows that another potential attack mitigation strategy would be to include the location in the NFC transaction as the attacks can only relay the data, not modify it. If this was done the reader could check that the location was within the expected range to be considered near the reader. The issue with this strategy is that the location is not always accurate, getting accurate location through GPS takes time and may not be possible if the reader is undercover.
\\
\indent Using NFC for my project will require investigation into the methods specified above to prevent relay attacks. This has certainly opened up a lot of potential work for me to ensure that the smart lock is easy to use but also secure from relay attacks. I'll be looking into how effective the distance bounding is, but if that proves ineffective or unreliable I will make it so the device has to be unlocked in order to respond to the NFC signal. I will be investigating and comparing these options to the security of a conventional key based lock system.

\section{TLS 1.2 Security}
TLS 1.2 is the most recent version of the transport layer protocol used to secure communications across a network to prevent eavesdropping and tampering. The security of TLS 1.2 is vital for the project as it is what will be used for all communications between the phone, the lock and the server.
\\
\indent\cite{Meyer2014} offers a fairly comprehensive study of SSL/TLS from the very beginning when SSL 1.0 was released. The study points out that SSL/TLS have had many vulnerabilities throughout the time that they have been commonplace in securing internet traffic. It also points out that there are lots of different implementations of the TLS standard, all of which have their own differences and bugs. OpenSSL is the most used implementation on the web at the time of the study based on the usage of web servers and browsers that use the implementations. Part of this study lists a lot of different attacks that have been possible against TLS or some of its implementations over time, all of which have now been resolved in the latest implementations of the TLS standard. The existence of this list does show however that TLS is not perfect, and neither are the implementations. As pointed out by \cite{Turner2014} a lot of the security is left down to the system administrators, whether that is ensuring that only the most secure algorithms are allowed to be used, or patching a security flaw by making a configuration change. TLS can be a very secure protocol if setup in the correct way using up-to-date implementations, but it can also be insecure if badly configured or not updated when exploits are found and patched.
\\
\indent This research has shown me that TLS 1.2 is secure and uses cryptographically secure and proven algorithms for all the encryption it forms, but must be configured to not use the old algorithms and must be kept up-to-date by the system administrators managing the systems using it. I conclude that TLS 1.2 is sufficient for securing the smart lock system from this research, more technical investigation will be done during implementation to ensure that the system is configured as securely as possible.

\section{Android Fingerprint Keystore Security}
This system will utilise the android fingerprint keystore functionality to store and protect the keys used to communicate with and unlock the smart lock. The security of this keystore is fundamental to the security of the app. \cite{Does2016} show in their study a few different attacks on the fingerprint authentication methods in Android and how you would go about doing them.
\\
\indent The first attack \cite{Does2016} show is trying to get the device to accept a fingerprint which isn't enrolled on the device. Due to the Android system itself not knowing about any of the fingerprints enrolled into the device and therefore having to query the Trusted Execution Environment (TEE) about whether the user provided the necessary authentication, this attack isn't feasible without replacing a core component of the Android system. The attack was possible, but only by replacing fingerprintd to return a value other than 0, which is the only value the system determines as a failure, every time it is invoked. To replace fingerprintd requires root access to the system, and the system will warn you on every boot that the /system directory has been modified and could be corrupt. This means the attack is not feasible as you would need to root the device in order to be able to perform this attack.
\\
\indent The second attack that \cite{Does2016} performed on the Android system was trying to replay AuthTokens from the Keystore to the Keymaster which would allow them to perform cryptographic operations authenticated by the replayed AuthToken. To do this they forced the system to give them the same AuthToken every time and then waited for the challenge ID to become the same twice. When the challenge ID was returned as the same twice, they were able to send the same AuthToken and authenticate with the Keymaster to perform cryptographic operations. Two key factors of this attack are that they retrieved the AuthToken from the device memory which would not be possible without root permissions in a normal case. Secondly, they forced the device to always return the same AuthToken, which is not normal behaviour as generally, the device will return a different AuthToken with every request, along with a different challenge ID, therefore avoiding this issue.
\\
\indent In conclusion, the study couldn't find any exploitable flaws with the fingerprint keystore security under normal circumstances, only when they had root access to the device and could, therefore, replace binaries with ones they had constructed themselves. For my use case, I assume that if someone can gain root access to the device, they already had the ability to unlock the device and therefore access the smart lock app to unlock the door.