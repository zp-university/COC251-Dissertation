% !TeX root = ../../main.tex
\chapter{Implementation}
\label{chap:implementation}
In this chapter I will give details of the implementation of the previously mentioned ideas and requirements. This will cover technical aspects of the implementation, problems experienced along the way, any tools used to aid development and discussion on what languages where used and why. This chapter will be split up into sections for each component of the system in order to discuss why certain choices were made for that specific platform.

\section{Tools Used}
\subsection{Version Control}
Version control is a very useful tool that is used throughout the software engineering industry as it allows much better management of source code and development in general. It contains a complete history of all changes that have been made to the project from the very beginning. This is very useful if you need to revert a change or diagnose a problem and find what change may have caused it.
\\
\indent There are a wide range of different Version Control Software's (VCS), however I decided to go with Git. Git is a very extensive and powerful tool, and is by far the most widely used VCS out there. Git can run locally and remotely so you can push your changes to a remote server so that others can see them and they are backed up. I use the open source free software Gitlab in order to host my own Git server so I used this for all my Git projects.
\\
\indent Alongside git itself I used the WebStorm and Android Studio git integration to help with committing code and pushing it to the remote server as both of those IDEs come with git integration built in. I also worked with git on the command line for pulling down the code on the devices that were running it.

\subsection{Integrated Development Environments (IDEs)}
IDEs are extremely useful tools that come bundled with a huge array of features all aiming to make developing easier and less error prone. They offer features such as auto-completion, auto-generation of certain code snippets, detection of potential errors, git integration, and much more.

\subsubsection*{WebStorm}
One of the IDEs used widely in this project was WebStorm as all of the APIs were written in Node.js which WebStorm has support for. Built-in support for Node.js, npm, git and more made the development process much smoother and faster. WebStorm is part of a much larger suite of IDEs made by JetBrains, they also make IntelliJ which is what AndroidStudio is based off of.

\subsubsection*{Android Studio}
Android Studio is the other IDE that I used in this project as it basically required if you want to write applications for Android. It comes with all the tools you need to compile and run android apps, including virtual android devices, a package manager for different versions of the android api, and many other great and useful tools. Installing and running all of this manually would take a huge amount of time, so using this tool vastly decreases the amount of time required for development and testing iterations.

\section{Central Server}
The first section of the project that I worked on was the central server. The reason for this was that every other part of the system would rely on this in some way, and so it would get in the way if not done early on in the project. The plan was that in the eyes of the other parts of the application, the central server would just be a REST API that they could call to send and retrieve data, and that data would get passed to the other parts of the application that required them.

\subsection{Programming Language}
For this section of the project I decided to go with Node.js as I have used it before to make REST APIs and it provides high performance and is very quick to develop in. I used it along with an API design tool called Swagger which allows you to create a specification for your entire API, automatically generating documentation for it. It can also automatically generate boilerplate code for the server side and client side applications to make development much faster, it's a really useful tool. One of the major dependencies used within my development was mongoose which is a library that allows you to connect to a MongoDB database and make database calls in an asynchronous way with minimal code. Node.js has the benefit of encouraging you to write everything in an asynchronous manner, so almost everything uses callbacks rather than conventional nested statements. This has a huge performance benefit as the entire application runs on multiple threads rather than just a single thread.

\subsection{Login API}
This was the first part of the server that needed to be written, and security was vital. I was originally going to go with a traditional OAuth system but that seemed over complex for what was needed, so I decided to go with a JSON Web Token (JWT) system. JWT's are stored by the client and are sent along with every request. They contain data that both the client and the server can read, as well as an encrypted hash of that same data using a key that is stored on the server. This hash means that the client can't modify the data and then send it to the server as then the encrypted hash wouldn't match. 
\\
\indent For my system, I wanted it to be possible to later build in a system where the user could invalidate any device that is logged into their account, forcing that device to login again. This would be useful if the device was stolen and the user didn't want the thief to be able to also get into their house. In order to accommodate this in the future, I designed the JWTs so that they contain an ID which is stored in the database which then links that ID to a User object. If the JWT needs to be invalidated, simply delete the ID from the database and the device will no longer be able to authenticate and will be forced to login again, requiring the username and password for the account.
\\
\indent This system is used for both users and devices except that a device doesn't require a username and password. A device can register itself with the system and receives a JWT in return, it gives the server it's UUID and name during this process. Once the device is registered, a user can then register it to their account.

\subsection{Device and User API}
The device API includes the login api section for devices, but also includes the ability for the device to query information about itself so that it can be continually updating based on what the user does. This is how it responds to lock and unlock requests for example. The User API has the ability to set the device that is linked to their account using the UUID of the device, this then binds the device to their account, and the user can then send lock/unlock requests to the device through the API and get the device status using the user API.

\subsection{Other Software}
The system runs Nginx which is a web server software, the Node.js API runs its own web server on localhost which the Nginx server forwards traffic to using a reverse proxy system. Docker is another piece of software that is used for this part of the system, this is due to the hardware it's running on being used for multiple other systems as it is my own personal home server so this keeps it isolated from the rest of the system. As mentioned above MongoDB was used for the database software, I chose this as it is fast and reliable and i've used it before on previous projects with amazing results, so it was an easy choice for me when deciding on a database software to use.

\section{Smart Lock Core - Raspberry Pi}
The next step of the project was to work on the APIs, hardware and software setup for the raspberry pi that would be at the core of the network. This stage was probably the second largest sticking point for the project as it required a lot of experimentation with the hardware to get it working in the exact way I required. There were also technical issues along the way that hindered my progress which I will detail below.

\subsection{Programming Language}
As with the central server, this was mostly API based and therefore all the code was again written in Node.js using mongoose for and MongoDB for the database to store whatever the Core would need locally. I chose MongoDB in this instance as the initial implementation of the software wouldn't need a database necessarily as it would only be storing a small amount of data, however for some of the optional requirements data would need to be stored locally and therefore it was better to set this up now rather than have to change everything later.

\subsection{Setting up a Wi-Fi Hotspot}
For this I decided to use the built-in Wi-Fi of the Raspberry Pi 3 (RPi) as it had the most information readily available about how to set it up to do this. This part required three pieces of software: dhcpcd; dnsmasq; and hostapd. The job of dhcpcd was to assign the Wi-Fi hardware an IP when the hotspot was brought online so that other devices could contact it once they connected to the Wi-Fi. Dnsmasq provided dhcp for the Wi-Fi hotspot itself so that people who connected to the Wi-Fi could get an IP so the RPi could respond to them. Finally hostapd was the piece of software that provided the hotspot itself, booting up the onboard Wi-Fi and broadcasting the network so that others could connect. These pieces of software worked together flawlessly, however the initial RPi that I was using had an issue with the built-in antenna so I spent a fair few hours trying to figure out why the range was only a couple of centimetres from the device which turned out to just be wasted time as a new RPi fixed the problem with the exact same setup. Once this was working, turning the Wi-Fi on and off required just two commands, which were controlled by the Node.js API whilst setup was in progress.

\subsection{Setup API}
The setup API is just a REST API in the same way that the core server is, however it only works on the IP of the Wi-Fi hotspot (192.168.4.1). This IP would be the same on every device so that the users phone knows what IP to use in order to access the setup API. The API has two endpoints, one which accepts the Wi-Fi details so it can connect to your home Wi-Fi and returns the UUID of the device, and another which tells you the status of the setup, either waiting, testing, failed, or completed. Once the status endpoint is called and returns completed, it waits 5 seconds and then shuts down the hotspot as the device is now successfully setup. The details endpoint however is a lot more complex, as after it has received the details from the user it does a lot of tasks in order to connect to the Wi-Fi network provided, check that it connects successfully, generate a new UUID for itself, send that to the central server to register the device and then respond to the users device with the new UUID it generated. The main issue I came up against with this system was having the Wi-Fi devices switch between hotspot mode and connecting to the access point automatically, as it would kick the users device off of the hotspot and cause issues. Instead of this I decided to add a second Wi-Fi adapter to the device via USB as it eased the complexity of this section slightly and reduced the likelihood of issues arising down the line.

\subsection{Core to Central Server Communication}
The core is constantly polling the server for any commands that it may receive. Originally I was planning to have a constantly open connection in the form of a websocket, however this uses more resources constantly on the server side and is less secure if somehow the TLS encryption of one session was cracked by an attacker. With new sessions, new nonces and encryption keys are generated and so even with the information from the last session, this new sessions security would not be lessened by the fact that the last session was decrypted. That said, the chances of a session getting decrypted are essentially impossible with current technology, so this is being overly secure if anything.

\section{Android Application}
The final section of this project is the Android App which is the only part that the user will see directly. Not being a creative type myself meant that creating the user interface for this app was fairly difficult as it needed to be easy to use and clear. I feel like i've done a good job with it overall and the setup process seems simple to me.

\subsection{Programming Language}
The Android Application was programmed in Java as that's the main language used for Android development and it's one I am very familiar with. I used a few dependencies to aid my development process, mostly in regard to making network requests. Retrofit is a library that provides a very simple way to turn a HTTP API into a Java interface, making network requests much simpler and cleaner to look at. RxJava is another dependency that I used as it works with Retrofit to make all of the network calls work in the background and fire callbacks when a response is received. With Android, working Asynchronously is very important as if you do requests on the UI thread, the entire interface will freeze which makes for a very bad user experience.

\subsection{Development}
The first section I worked on was the login and sign up screens, ensuring that they would appear as soon as the user opens the application for the first time. I went with a minimal approach, only asking for the information that is required and made moving between the signup and login screens just a single tap or swipe away. The next part I worked on was the device setup screen which is by far the most complex part of the app. This page alone requires controlling what Wi-Fi the device is connected to, making API calls to the core raspberry pi and the central server and waiting until the right time to move on to the next screen. This screen gave me a lot of problems due to the complexity of changing Wi-Fi networks whilst trying to make requests to different APIs. The final screen that I worked on was the home screen which allows you to lock or unlock the smart lock. Originally there was planned to be a little more functionality but due to time constraints I had to build a working demo and cut the more fancy features that I had planned.

\section{Improvements}
Here I will list any improvements I would make if I were to approach this project again with this experience in mind. I've learnt a lot from this project and there are certain things that I would change to make the project work better or improve overall security of the product.

\subsection{Wi-Fi Hotspot Security}
Whilst programming the Wi-Fi hotspot section of the android application, I realised that there was no possible way to verify that the device I was connecting to was the device that I had in front of me using the current hardware I had. By the time I realised this I wouldn't have had time to order the extra parts required to get it ready in time for a demo, however I came up with a solution to the problem. The issue lies within how certificate trust works, with a certificate authority validating that you own a domain and then providing a public and private key pair to you. In a private network like the one created with a hotspot, this is not possible, and creating your own self-signed certificate means that the android application can't verify which device it is connected to. This would allow someone who was waiting for you to setup your device to create another Wi-Fi hotspot with the same name and have you connect to them. By doing this they could get access to your Wi-Fi password as you wouldn't realise it wasn't the same device that you see in front of you. In order for the user to confirm this, I would want to add a small screen to the device that could display a short code or QR code that the phone could scan as this direct method of communication can be verified. The code would be a hash of the https certificates public key and so could be verified by the phone to ensure they match, and if they do you're connected to the right device. This is a very unlikely scenario anyway as it requires an attacker to be waiting outside your home for you to setup the smartlock, and realistically they have a very small window to mimic the smart lock Wi-Fi and have the user connect to them instead.

\subsection{Another Improvement}
\improvement{Write another improvement}

\section{Compared to other locks}
Based on the issues I mentioned with other known locks in my problem statement, I will now evaluate what I have done to remove those security risks in my smart lock.

\subsection{August Firmware Key}
This problem can't happen on my system due to the fact that no master key exists. Everything is based around the user accounts themselves and communication with the central server meaning you have to be able to authenticate with the server to unlock the smart lock. On top of that, every time the smart lock is setup, it generates a new UUID which is used to identify itself on the server, so there is no chance that another user could register the lock to their account in the few second window where the lock is unregistered to an account on the central server.

\subsection{Removing Users whilst Offline}
With the currently developed system, the lock doesn't operate when offline as that functionality wasn't implemented. However, with the plan for offline functionality that I outlined in the design section, this lock would not be vulnerable to this attack as specific offline keys would have to be distributed by the lock owner, or another person who was trusted with the responsibility to distribute offline keys. Once the system comes back online then the keys immediately stop working and the lock must then be operated using the normal user authentication methods. In the case of other locks, communication with the internet could be blocked to the smartlock and then an attacker could use a compromised device to access the lock as the lock owner would be unable to remove the compromised device from the lock without being at home.